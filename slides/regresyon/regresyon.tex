%%%%%%%%%%%%%%%%%%%%%%%%%%%%%%%%%%%%%%%%%%%%%%%%%%%%%%%%
%%  Template for beamer presentation in rmarkdown
%%  Modified from the original default.latex template
%%  file of pandoc
%%  --Murat Taşdemir
%%%%%%%%%%%%%%%%%%%%%%%%%%%%%%%%%%%%%%%%%%%%%%%%%%%%%%%%
% Options for packages loaded elsewhere
\PassOptionsToPackage{unicode}{hyperref}
\PassOptionsToPackage{hyphens}{url}
%
\documentclass[
  12pt,
  ignorenonframetext,
  fleqn,
  aspectratio=43]{beamer}
   
\usepackage{pgfpages}
\setbeamertemplate{caption}[numbered]
\setbeamertemplate{caption label separator}{: }
\setbeamercolor{caption name}{fg=normal text.fg}
\beamertemplatenavigationsymbolsempty
% Prevent slide breaks in the middle of a paragraph
\widowpenalties 1 10000
\raggedbottom
\setbeamertemplate{part page}{
  \centering
  \begin{beamercolorbox}[sep=16pt,center]{part title}
    \usebeamerfont{part title}\insertpart\par
  \end{beamercolorbox}
}
\setbeamertemplate{section page}{
  \centering
  \begin{beamercolorbox}[sep=12pt,center]{part title}
    \usebeamerfont{section title}\insertsection\par
  \end{beamercolorbox}
}
\setbeamertemplate{subsection page}{
  \centering
  \begin{beamercolorbox}[sep=8pt,center]{part title}
    \usebeamerfont{subsection title}\insertsubsection\par
  \end{beamercolorbox}
}
\AtBeginPart{
  \frame{\partpage}
}
\AtBeginSection{
  \ifbibliography
  \else
    \frame{\sectionpage}
  \fi
}
\AtBeginSubsection{
  \frame{\subsectionpage}
}

\usepackage{lmodern}
\usepackage{amssymb,amsmath}
\usepackage{ifxetex,ifluatex}
\ifnum 0\ifxetex 1\fi\ifluatex 1\fi=0 % if pdftex
  \usepackage[T1]{fontenc}
  \usepackage[utf8]{inputenc}
  \usepackage{textcomp} % provide euro and other symbols
\else % if luatex or xetex
  \usepackage{unicode-math}
  \defaultfontfeatures{Scale=MatchLowercase}
  \defaultfontfeatures[\rmfamily]{Ligatures=TeX,Scale=1}
\fi
\usetheme[]{metropolis}
% Use upquote if available, for straight quotes in verbatim environments
\IfFileExists{upquote.sty}{\usepackage{upquote}}{}
\IfFileExists{microtype.sty}{% use microtype if available
  \usepackage[]{microtype}
  \UseMicrotypeSet[protrusion]{basicmath} % disable protrusion for tt fonts
}{}
\makeatletter
\@ifundefined{KOMAClassName}{% if non-KOMA class
  \IfFileExists{parskip.sty}{%
    \usepackage{parskip}
  }{% else
    \setlength{\parindent}{0pt}
    \setlength{\parskip}{6pt plus 2pt minus 1pt}}
}{% if KOMA class
  \KOMAoptions{parskip=half}}
\makeatother
\usepackage{xcolor}
\IfFileExists{xurl.sty}{\usepackage{xurl}}{} % add URL line breaks if available
\IfFileExists{bookmark.sty}{\usepackage{bookmark}}{\usepackage{hyperref}}
\hypersetup{
  pdftitle={Regresyon},
  pdfauthor={Murat Taşdemir},
  hidelinks,
  pdfcreator={LaTeX via pandoc}}
\urlstyle{same} % disable monospaced font for URLs
\newif\ifbibliography
\setlength{\emergencystretch}{3em} % prevent overfull lines
\providecommand{\tightlist}{%
  \setlength{\itemsep}{0pt}\setlength{\parskip}{0pt}}
\setcounter{secnumdepth}{-\maxdimen} % remove section numbering
%%%%%%%%%%%%%%%%%%%%%%%%%%%%%%%%%%%%%%%%%%%%%%%%%%%%%%%%%%%%%%
%%  Some beamer customization
%%%%%%%%%%%%%%%%%%%%%%%%%%%%%%%%%%%%%%%%%%%%%%%%%%%%%%%%%%%%%%
	
\setbeamertemplate{headline}{}
\setbeamertemplate{footline}
{
 \leavevmode%
  \hbox{%
 \begin{beamercolorbox}[wd=.5\paperwidth,ht=2.5ex,dp=1.125ex,leftskip=.3cm,rightskip=.3cm]{author in head/foot}%
    \usebeamerfont{author in head/foot}\insertshortauthor
  \end{beamercolorbox}%
 \begin{beamercolorbox}[wd=.5\paperwidth,ht=2.25ex,dp=1ex,right]{date in head/foot}%
    \usebeamerfont{date in head/foot}\insertshortdate{}\hspace*{2em}
    \insertframenumber{} / \inserttotalframenumber\hspace*{2ex} 
  \end{beamercolorbox}}%
  \vskip0pt%
} 



\usepackage{polyglossia}
\setdefaultlanguage{turkish}
\usepackage{booktabs}
\usepackage{caption}
\usepackage{amsmath}
\usepackage{setspace}
\usepackage{float}
\usepackage{multirow}
\usefonttheme{professionalfonts}
\usepackage{mathtools}
\usepackage{unicode-math}
\setmathfont{TeX Gyre Pagella Math}
\newcommand*\rot{\rotatebox{90}}
\AtBeginSection{}
\let\olditem\item
\renewcommand{\item}{\olditem\vspace{\fill}}
\useinnertheme{circles}
\definecolor{mLightBrown}{HTML}{EB811B}
\setbeamercolor{itemize item}{fg=mLightBrown}
\setbeamertemplate{enumerate items}{\bf\insertenumlabel.}
\setbeamercolor{item projected}{bg=mLightBrown,fg=black}
\metroset{progressbar=none, numbering=fraction}

\title{Regresyon}
\subtitle{İKT 533}
\author[MT]{Murat Taşdemir}
\date{15 Kasım 2020}
\institute[İMÜ]{İMÜ}

\begin{document}
\frame{\titlepage}

\begin{frame}{HANGİ OKULA GİTMELİ?}
\protect\hypertarget{hangi-okula-gitmeli}{}
\begin{itemize}
\tightlist
\item
  Özel
\item
  Devlet
\end{itemize}
\end{frame}

\begin{frame}{Tesadüfileştirme Yerine Regresyon}
\protect\hypertarget{tesaduxfcfileux15ftirme-yerine-regresyon}{}
\begin{itemize}
\item
  Her zaman tasadüfileştirilmiş deney yapma şansımız yok.
\item
  Regresyon kullanarak nedensel etkiyi tahmin edebiliriz (dikkat!)
\item
  İki alternatif için de potansiyel sonuçları tanımlayalım.
\item
  Özel okullar (üniversite) verilen paraya değer mi? (ABD bağlamında)

  \begin{itemize}
  \tightlist
  \item
    \(Y_{1i}\): özel üniversiteye giden (\(P_i=1\)) \(i\) bireyinin
    ücreti
  \item
    \(Y_{0i}\): \(i\) bireyinin karşıolgusal (\(P_i=0\)) sonucu
  \end{itemize}
\end{itemize}
\end{frame}

\begin{frame}{Yapısal (gerçek nedensel) ilişki}
\protect\hypertarget{yapux131sal-geruxe7ek-nedensel-iliux15fki}{}
\begin{itemize}
\item
  Özel \(Y_{0i}\) lerin (ortalamada) daha yüksek olduğunu varsayalım.

  \begin{itemize}
  \tightlist
  \item
    \alert{Dikkat edin} bu \(Y_i\)'nin \(P_i\)'ye bağlı olarak değiştiği
    anlamına gelir.
  \item
    Regresyon kullanarak seçilim yanlılığınız azaltabilir, hatta yok
    edebiliriz
  \end{itemize}
\item
  \(Y_{0i}=\alpha_i+\eta_i\) ve \(Y_{1i}-Y_{0i}=\beta\) olsun

  \begin{itemize}
  \tightlist
  \item
    \(\alpha_i\): devlet okuluna giden bireyin ortalama ücreti
  \item
    \(\eta_i\): tesadüfi kısım
  \item
    \(\beta\): özel üniversiteye gitmenin ücretteki etkisi (müdahale
    etkisi)
  \end{itemize}
\end{itemize}
\end{frame}

\begin{frame}{Regresyon ve Koşullu Ortalama Varsayımı (CIA)}
\protect\hypertarget{regresyon-ve-koux15fullu-ortalama-varsayux131mux131-cia}{}
\begin{itemize}
\tightlist
\item
  Tesadüfi kısımın beklenen değerinin sıfır olduğunu biliyoruz:
  \(E[\eta_i]=0\).
\item
  Bu gerçek \(E[\eta_i|P_i]\) koşullu beklenen değerinin sıfır olmadığı
  anlamına \alert{gelmez.}.
\item
  Aksine okul seçimini etkileyen başka değişkenler olduğu için
  \(E[\eta_i|P_i]\neq 0\) olduğunu biliyoruz
\item
  \(X_i\) kontrol değişkenlerinin aşağıdaki \emph{koşullu bağımsızlık
  varsayımı}nı (CIA) sağladığını varsayacağız:
  \[E[\eta_i|P_i,X_i]=E[\eta_i|X_i]=\gamma'X_i\]
\end{itemize}
\end{frame}

\begin{frame}{Regresyon}
\protect\hypertarget{regresyon}{}
\begin{itemize}
\item
  Buradan \[Y_i=\alpha+\gamma'X_i+\beta P_i+u_i\]
\item
  \(\beta\): nedensel etki
\item
  \(\gamma\): ilgilenmiyoruz
\end{itemize}
\end{frame}

\end{document}
