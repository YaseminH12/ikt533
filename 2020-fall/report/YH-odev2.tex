\documentclass{article}
\usepackage{geometry}
\usepackage{ragged2e}
\usepackage{graphicx}
\usepackage{wrapfig}
\usepackage{csvsimple}
\usepackage[utf8]{inputenc}
\usepackage{amsmath}
\usepackage{placeins}

\graphicspath{ {./graphics/} }
\geometry{a4paper, top=2cm, bottom=1cm, hmargin=2.3cm,  includehead, includefoot}

\begin{document}
    \title{\textbf{Emek Ekonomisi Ödev-2 Çözümleri} \par 
     Ceritoğlu vd. (2017) Makale Replikasyonu}
    \author{Yasemin Hayırlı}
    \date{14 Ocak 2021}
    \maketitle

\subsection*{Makale Özeti ve Tanımlayıcı İstatistikler}
\setlength{\parindent}{0em}
\setlength{\parskip}{0.3em}

    \subparagraph{Makale Özeti} 
    \begin{justify}
    \setlength{\parindent}{0em}
        2011 yılında başlayan Arap Baharı, milyonlarca Suriyelinin komşu ülkelere 
        iltica etmesine yol açmıştır. Özellikle MENA bölgesi ve Türkiye, Suriyeli 
        mültecilerin büyük kısmına ev sahipliği yapmıştır. Türkiye özelinde Suriyeli 
        mültecilerin farklı bölgelere göçü, doğal bir deney olarak ele alınmış ve bu 
        deneyden yararlanarak Ceritoğlu vd.(2017) Suriyeli mültecilerin Türkiye'de 
        yerel emek piyasasında yarattığı etkiyi farkların farkı metodu ile
        belirlemeye çalışmışlardır. 

        Suriyeli mültecilerin Türkiye'ye kitlesel akını, 2012 yılında başlamıştır.
        Bu nedenle çalışmada müdahale etkisinin başlangıç tarihi 1 Ocak 2012 olarak
        alınmış ve bu tarih referans alınarak müdahale öncesi ve müdahale sonrası
        dönem belirlenmiştir. Bu doğrultuda 2010-2011 yılları arasındaki dönem müdahale 
        öncesindeki periyodu, 2012-2013 yılları arasındaki dönem müdahale sonrasındaki 
        periyodu oluşturmaktadır. Müdahalenin yapıldığı bölgeler ise mültecilerin toplam
        nüfusa göre oranına bakılarak belirlenmiştir. Özellikle Türkiye hükümeti tarafından
        kurulan mülteci kamplarının yer aldığı iller ve çevresine odaklanmışlardır.
        Suriyeli mültecilerin toplam nüfusa oranının \%2'den yüksek olduğu bölgeler, 
        deney grubunu; \%2'den daha az olduğu bölgeler ise kontrol grubunu oluşturmaktadır. 
        Ayrıca deney ve kontrol grupları seçilirken bu iki grubun kültürel, ekonomik,
        sosyo-demografik açıdan benzer olmasına dikkat edilmiştir. Müdahale grubu, NUTS2
        sınıflamasına göre Adana, Hatay, Gaziantep, Şanlıurfa ve Mardin bögelerini;
        deney grubu ise Erzurum, Ağrı, Malatya ve Van bölgelerini kapsamaktadır.

        Ceritoğlu vd.(2017) çalışmalarında veri seti olarak Hanehalkı İşgücü Anketlerinin
        2010, 2011, 2012 ve 2013 yıllarına ait gözlemlerini kullanmışlardır. Seçilen örneklem
        sadece 15-64 yaş aralığındaki yerli emek gücünü kapsamaktadır. Bu doğrultuda 
        yurtdışında doğup 2010 yılından sonra Türkiye'ye giriş yapan bireyler analize dahil edilmemiştir.
        Örneklem bu hali ile \textit{354.023} adet gözlem içermektedir.
    \end{justify}


    \newpage

    \subparagraph{Tanımlayıcı İstatistikler}
    \begin{justify}
        Tablo 1 deney ve kontrol gruplarında yer alan bireylerin sosyal ve demografik 
        özelliklerini göstermektedir. Gruplara ait özellikler birbirine oldukça yakındır.
        Ayrıca gruplar bazında 2010-2013 yılları arasında büyük bir değişime de rastlanılmamıştır.
        Bu açıdan farkların farkı metodunun kullanılabilmesi için gerekli olan 
        \textit{"Paralel Trendler Varsayımı"} gerçekleşmiş olmaktadır. 

        \FloatBarrier
        \begin{table}[h]
            \centering
            \caption{Yerel Nüfus İçin Demografik İstatistikler}
            \begin{tabular}{|llllll|}
            \hline
                 & \multicolumn{2}{|c|}{Müdahale Öncesi} & \multicolumn{2}{|c|}{Müdahale Sonrası} &\\
                 & 2010 & 2011 & 2012 & 2013 & Toplam \\ \hline
                \multicolumn{6}{|c|}{Deney Grubu}\\ \hline
                Erkek  & 0,481 & 0,482 & 0,48 & 0,48 & 0,481 \\ 
                Yaş & 34,5 & 34,9 & 35,2 & 35,6 & 35 \\ 
                Evli & 0,645 & 0,64 & 0,635 & 0,638 & 0,64 \\ 
                Lise ve Üstü & 0,112 & 0,123 & 0,131 & 0,141 & 0,234 \\ 
                Şehirleşme & 0,729 & 0,743 & 0,74 & 0,752 & 0,741 \\ 
                Gözlem Sayısı & 58143 & 56382 & 56167 & 54656 & 225348 \\ \hline 
                \multicolumn{6}{|c|}{Kontrol Grubu}\\ \hline
                Erkek  & 0,479 & 0,489 & 0,491 & 0,49 & 0,487 \\
                Yaş & 34,3 & 34,2 & 34,5 & 34,4 & 34.2 \\ 
                Evli & 0,643 & 0,635 & 0,633 & 0,628 & 0,635 \\ 
                Lise ve Üstü & 0,116 & 0,128 & 0,144 & 0,139 & 0,239 \\ 
                Şehirleşme & 0,532 & 0,509 & 0,512 & 0,525 & 0,52 \\ 
                Gözlem Sayısı & 33646 & 32614 & 31127 & 31288 & 128675 \\ \hline
            \end{tabular}
        \end{table}
        \FloatBarrier

        Yerli nüfusun işgücü istatistikleri ise Tablo 2'de gösterilmiştir. Müdahale ve kontrol 
        gruplarının birbirinden çok farklı olmadığı görülmektedir.



        \FloatBarrier
        \begin{table}[h]
            \centering
            \caption{Yerel Nüfus İçin İşgücü İstatistikleri}
            \begin{tabular}{|cccccc|}
            \hline
                 & \multicolumn{2}{|c|}{Müdahale Öncesi} & \multicolumn{2}{|c|}{Müdahale Sonrası} &\\
                 & 2010 & 2011 & 2012 & 2013 & Toplam \\ \hline
                \multicolumn{6}{|c|}{Deney Grubu}\\ \hline  
                Toplam İstihdam Oranı  & 0,406 & 0,409 & 0,396 & 0,406 & 0,404 \\  
                Kayıtlı İstihdam Oranı & 0,172 & 0,181 & 0,192 & 0,211 & 0,189 \\ 
                Kayıtsız İstihdam Oranı & 0,234 & 0,227 & 0,204 & 0,195 & 0,215 \\ 
                İşsizlik Oranı & 0,068 & 0,054 & 0,051 & 0,064 & 0,059 \\ 
                İş Gücüne Katılım Oranı & 0,474 & 0,462 & 0,447 & 0,470 & 0,463 \\
                İşten Ayrılma Olasılığı & 0,015 & 0,012 & 0,013 & 0,018 & 0,014 \\ \hline
                \multicolumn{6}{|c|}{Kontrol Grubu}\\ \hline
                Toplam İstihdam Oranı  & 0,446 & 0,468 & 0,479 & 0,500 & 0,473 \\
                Kayıtlı İstihdam Oranı & 0,156 & 0,171 & 0,184 & 0,191 & 0,175 \\ 
                Kayıtsız İstihdam Oranı & 0,290 & 0,298 & 0,295 & 0,309 & 0,298\\ 
                İşsizlik Oranı & 0,062 & 0,054 & 0,043 & 0,046 & 0,062 \\ 
                İş Gücüne Katılım Oranı & 0,508 & 0,522 & 0,522 & 0,474 & 0,524 \\
                İşten Ayrılma Olasılığı & 0,012 & 0,011 & 0,011 & 0,012 & 0,011 \\ \hline
            \end{tabular}
        \end{table}
        \FloatBarrier
    \end{justify}
    
    \newpage

    \subsection*{Ekonometrik Yöntem ve Analiz Sonuçları}
    \setlength{\parindent}{0em}
    \setlength{\parskip}{0.3em}

    \subparagraph{Ekonometrik Yöntem} 
    \begin{justify}
    \setlength{\parindent}{0em}
    Farkların farkı yöntemi, müdahaleye maruz kalan grup ile müdahaleden etkilenmeyen grup 
    arasında zaman içinde meydana gelen değişiklikleri karşılaştırmayı sağlayan yarı deneysel bir 
    analiz metodur. Ceritoğlu vd.(2017) çalışmalarında Suriyeli mültecilerin Türkiye'ye zoraki 
    göçünü yarı deney olarak ele almışlar ve 2010-2013 yılı arasında Türkiye'de emek piyasasında
    nasıl değişimler yaşandığını analiz etmişlerdir. Müdahale grubu ve kontrol grubu seçilirken
    Suriyeli mültecilerin toplam nüfusa oranı baz alınmıştır. Buna göre Suriyeli mültecilerin 
    toplam nüfusa oranının \%2'nin yukarısında olduğu bölgeler müdahale grubunu, \%2'nin aşağısında
    olan bölgeler ise kontrol grubunu oluşturmaktadır. Ayrıca seçilen gruplar ekonomik, sosyo-kültürel,
    demografik vb. açılardan birbirine benzerdir. Mülteci etkisi aşağıda yer alan regresyon modeli 
    aracılığıyla tahmin edilmiştir.

    \[ Y_{ijt} = \alpha + \beta (R_i x T_i) + \theta^t X_{ijt} +
    \sum_{j=1}^{8}\beta_j BOLGE_j + \sum_{i=2010}^{2012}\beta_i YIL_{i} + \kappa Z_{jt} + \epsilon_{ijt}  \]
    
    Mülteci etkisi, müdahalenin yapıldığı yıl olan 2012 ve 2013'te deney (müdahale) grubu olarak seçilen 
    bölgede ortaya çıkmaktadır.
    Bu etkinin belirlenmesi için yıllara ve bölgelere göre müdahalenin gerçekleşip gerçekleşmediğini 
    gösteren kukla değişkenler oluşturulmuştur. Modelde yer alan $T$ kukla değişkeni, 2010 ve 2011
    yılları için 0, 2012 ve 2013 yılları için ise 1 değerini; $R$ kukla değişkeni ise 
    müdahalenin uygulandığı bölgeler için 1, kontrol grubu olarak seçilen bölgeler için 0 değerini
    almaktadır. Bu değişkenlerin aynı anda 1 değerini aldığını gösteren etkileşim teriminin katsayısı 
    $\beta$, mülteci etkisini vermektedir.
    
    Regresyon modelinde yer alan $Y_{ijt}$ bağımlı değişkendir. Çalışmada bağımlı değişken olarak
    kayıtlı ve kayıtsız istihdam oranı, iş gücüne katılım oranı, işsizlerin oranı, iş ayrımı ve iş 
    bulma olasılıkları ve aylık ortalama reel ücret geliri kullanılmıştır. Ayrıca çalışmada mülteci 
    etkisinin yaşlara göre nasıl değiştiği de tahmin edilmiştir.
    
    Modelde yer alan $i,j$ ve $t$ endeksleri sırasıyla bireyleri, bölgeleri ve yılları temsil etmektedir.
    $X_{ijt}$ ise bireylerin gözlemlenen özelliklerinden oluşan vektördür. Bu bağlamda modele bireylerin
    yaşları, medeni durumları ve eğitim seviyeleri ve işgücüne dair bazı özellikleri dahil edilmiştir. 
    Ayrıca müdahale grubu, kontrol grubuna göre biraz daha fazla şehirleştiği için bu durum kontrol edilmek 
    istenmiş, bu amaçla modele bölgelerin şehirleşme seviyelerini nüfus bazında gösteren değişken modele eklenmiştir.

    Müdahale bölgesinde ekonomik aktivitenin bir kısmının Suriye ile yapılan ticarete bağlı olması ihtimali 
    göz önünde bulundurularak, modele ayrıca bölgelerin toplam dış ticaret hacmini gösteren $Z_{jt}$
    değişkeni dahil edilmiştir. Böylece Suriye iç savaşı nedeniyle müdahale bölgesinde yaşanan ticari daralma
    kontrol edilmiştir. 

    Modelde yıllara ve bölgelere göre sabit etkileri konrol edebilmek için düzeltilmiş yıl ve bölge 
    değişkenleri kullanılmıştır. Yıl ve bölgelere göre oluşan sabit etkilerin kontrol edilmesi amacıyla
    söz konusu değişkenlerin değerleri için ayrı ayrı kukla değişkenler oluşturulmuştur. Değişkenlerin kendi
    aralarında karşılaştırma yapılabilmesi için her değişkenin bir değeri sabit alınmıştır. 

    \end{justify}

    \newpage
    \subparagraph*{Analiz Sonuçları}
    \begin{justify}
    \setlength{\parindent}{0em}

    Bu bölümde ilk olarak Suriyeli mültecilerin kayıtdışı emek gücü üzerinde nasıl bir etki
    yarattığı analiz edilmiştir. Tablo 3 yerli kayıtdışı istihdam oranının cinsiyet ve eğitim seviyelerine
    göre nasıl değiştiğini göstermektedir. Analiz sonuçlarına göre toplam ticaret göz ardı edildiğinde
    kayıtdışı istihdam oranı \%2 civarında azalmaktadır. Bu sonuç toplam ticaret modele eklendiği 
    zaman da çok değişmemektedir. Kayıtlı istihdam oranının cinsiyet bazında nasıl değiştiğine dair 
    sonuçlar da Tablo 3'ten görülebilir. Buna göre Suriyeli mülteciler, kayıtdışı istihdamda kadınları,
    erkeklere göre daha fazla etkilemiştir. Erkek kayıtdışı istihdam \%1,8 
    azalma gösterirken, bu durum kadınlarda \%2,6 civarındadır. Eğitim gruplarına göre yapılan 
    analizin sonuçlarına göre ise düşük eğitimli bireylerin kayıtdışı istihdam oranlarının \%3 civarında
    azaldığı görülmektedir. Yüksek eğitim grubunda ise sonuçlar anlamsızdır. 
    
    Suriyeli mültecilerin Türkiye'de çalışma izni bulunmamaktadır ve ucuz emek arzı sağlamalarından dolayı niteliksiz 
    işgücü ile ikame edilmişlerdir. Dolayısıyla mülteci etkisi en fazla kayıtdışı sektörlerde çalışan bireyleri 
    etkilemektedir.
    
        \FloatBarrier
        \begin{table}[h]
            \centering
            \caption{Kayıtsız İstihdam Oranı İçin Regresyon Sonuçları}
            \begin{tabular}{|ccccccc|}
                \hline
                     & Toplam & Toplam & Erkek & Kadın & Düşük E. & Yüksek E. \\ \hline
                    Mülteci Etkisi (RXT) & -0.0227*** & -0.0224*** & -0.0185*** & -0.0264*** & -0.0337*** & 0.00706 \\ 
                     & (0.00280) & (0.00280) & (0.00442) & (0.00344) & (0.00338) & (0.00456) \\ 
                    Log(Toplam Ticaret) & - & + & + & + & + & + \\ 
                    Düzeltilmiş Yıl Etkisi & + & + & + & + & + & + \\ 
                    Düzeltilmiş Bölge Etkisi & + & + & + & + & + & + \\ 
                    Diğer Kontroller & + & + & + & + & + & + \\
                    Sabit Terim & 0.0798*** & 0.426** & 0.913*** & 0.119 & 0.523** & -0.0588 \\ 
                     & (0.00255) & (0.150) & (0.243) & (0.177) & (0.181) & (0.247) \\ 
                    R & 0.129 & 0.129 & 0.101 & 0.125 & 0.129 & 0.053 \\ 
                    Gözlem Sayısı & 354.023 & 354.023 & 171.017 & 183.006 & 270.608 & 83.415 \\ 
                    \multicolumn{7}{|c|}{\scriptsize Güçlü standart hatalar parantez içinde verilmiştir.   * p<0.05, ** p<0.01, *** p<0.001}\\ \hline
                    \multicolumn{7}{|c|}{\scriptsize Kontrol değişkeni olarak cinsiyet, medeni durum, yaş ve eğitim kukla değişkenleri, yaş ve eğitim değişkenlerinin etkileşim terimleri} \\ 
                    \multicolumn{7}{|c|}{\scriptsize ayrıca bölgenin şehirleşme durumunu gösteren kukla değişken kullanılmıştır.} \\ \hline
                    \multicolumn{7}{|c|}{\scriptsize Yüksek eğitim lise ve üstü eğitimi, düşük eğitim ise lise altı eğitimi ifade etmektedir.} \\ \hline

                \end{tabular}
        \end{table}
        \FloatBarrier

        Tablo 4, Suriyeli mülteci akımından dolayı bireylerin işgücüne katılımının nasıl değiştiğini
        göstermektedir. Yerli emek piyasasında yer alan bireylerin işgücüne katılım oranı \%1 civarında 
        azalmıştır. Bu oran kadınlarda \%2 civarında iken, erkeklerde sonuçlar anlamsız çıkmıştır. 
        Ayrıca düşük eğitimli bireylerin işgücüne katılım oranlarında \%1,7 civarında bir düşüş söz konusudur.
        Yüksek eğitim grubunda yer alan bireylerde ise herhangi bir etki görülmemiştir.


        \FloatBarrier
        \begin{table}[h]
            \centering
            \caption{İşgücüne Katılım Oranı İçin Regresyon Sonuçları}
            \begin{tabular}{|ccccccc|}
                \hline
                     & Toplam & Toplam & Erkek & Kadın & Düşük E. & Yüksek E. \\ \hline
                     Mülteci Etkisi (RXT) & -0.0110*** & -0.0117*** & 0.00431 & -0.0279*** & -0.0175*** & 0.00338    \\ 
                     & (0.00283) & (0.00283) & (0.00379) & (0.00397) & (0.00323) & (0.00576)    \\ 
                     Log(Toplam Ticaret) & - & + & + & + & + & + \\ 
                    Düzeltilmiş Yıl Etkisi & + & + & + & + & + & + \\ 
                    Düzeltilmiş Bölge Etkisi & + & + & + & + & + & + \\ 
                    Diğer Kontroller & + & + & + & + & + & + \\
                    Sabit Terim & 0.0412*** & -0.808*** & -0.992*** & -0.176 & -0.521** & -1.849*** \\ 
                    & (0.00274) & (0.154) & (0.216) & (0.206) & (0.175) & (0.324)    \\ 
                    R & 0.355 & 0.355 & 0.272 & 0.141 & 0.353 & 0.294    \\ 
                    Gözlem Sayısı & 354.023 & 354.023 & 171.017 & 183.006 & 270.608 & 83.415 \\ \hline
                    \multicolumn{7}{|c|}{\scriptsize Güçlü standart hatalar parantez içinde verilmiştir.   * p<0.05, ** p<0.01, *** p<0.001}\\ \hline
                    \multicolumn{7}{|c|}{\scriptsize Kontrol değişkeni olarak cinsiyet, medeni durum, yaş ve eğitim kukla değişkenleri, yaş ve eğitim değişkenlerinin etkileşim terimleri} \\ 
                    \multicolumn{7}{|c|}{\scriptsize ayrıca bölgenin şehirleşme durumunu gösteren kukla değişken kullanılmıştır.} \\ \hline
                    \multicolumn{7}{|c|}{\scriptsize Yüksek eğitim lise ve üstü eğitimi, düşük eğitim ise lise altı eğitimi ifade etmektedir.} \\ \hline

                \end{tabular}
        \end{table}
        \FloatBarrier

        \newpage

        Yerli emek piyasasında Suriyeli mültecilerin işsizlik oranlarında yarattığı etki Tablo 5'te
        gösterilmektedir. Buna göre müdahale grubunda işsizlik oranları toplamda \%0,08 artmıştır.
        Erkeklerde bu oran \%0,1 civarında iken kadınlarda ortaya çıkan sonuçlar anlamsızdır. Bu sonuç kadınların
        büyük ölçüde emek piyasasından çıkması ile açıklanabilir. Düşük eğitim grubunda ise işsizlik 
        oranı \%0,07, yüksek eğitim grubunda \%0,08 civarında artmıştır
        \footnote{Makalede yüksek eğitimli işgücü grubunda işsizlik oranları istatiksel olarak anlamsız çıkmıştır.}. 
        Suriyeli mültecilerin, niteliksiz işgücü ile ikame edilmesinden dolayı düşük eğitimli işgücünde işsizlik oranının 
        artması beklenir bir sonuçtur.

        \FloatBarrier
        \begin{table}[h]
            \centering
            \caption{İşsizlik Oranı İçin Regresyon Sonuçları}
            \begin{tabular}{|ccccccc|}
                \hline
                     & Toplam & Toplam & Erkek & Kadın & Düşük E. & Yüksek E. \\ \hline
                     Mülteci Etkisi (RXT) & 0.00867*** & 0.00810*** & 0.0150*** & 0.00154 & 0.00774*** & 0.00836*   \\ 
                     & (0.00141) & (0.00140) & (0.00255) & (0.00126) & (0.00149) & (0.00343)    \\ 
                    Log(Toplam Ticaret) & - & + & + & + & + & + \\ 
                    Düzeltilmiş Yıl Etkisi & + & + & + & + & + & + \\ 
                    Düzeltilmiş Bölge Etkisi & + & + & + & + & + & + \\ 
                    Diğer Kontroller & + & + & + & + & + & + \\
                    Sabit Terim & 0.0305*** & -0.669*** & -1.213*** & -0.139* & -0.667*** & -0.770*** \\ 
                    & (0.00139) & (0.0739) & (0.137) & (0.0613) & (0.0792) & (0.179)    \\
                    R & 0.034 & 0.034 & 0.021 & 0.044 & 0.040 & 0.032    \\ 
                    Gözlem Sayısı & 354.023 & 354.023 & 171.017 & 183.006 & 270.608 & 83.415 \\ \hline
                    \multicolumn{7}{|c|}{\scriptsize Güçlü standart hatalar parantez içinde verilmiştir.   * p<0.05, ** p<0.01, *** p<0.001}\\ \hline
                    \multicolumn{7}{|c|}{\scriptsize Kontrol değişkeni olarak cinsiyet, medeni durum, yaş ve eğitim kukla değişkenleri, yaş ve eğitim değişkenlerinin etkileşim terimleri} \\ 
                    \multicolumn{7}{|c|}{\scriptsize ayrıca bölgenin şehirleşme durumunu gösteren kukla değişken kullanılmıştır.} \\ \hline
                    \multicolumn{7}{|c|}{\scriptsize Yüksek eğitim lise ve üstü eğitimi, düşük eğitim ise lise altı eğitimi ifade etmektedir.} \\ \hline

                \end{tabular}
        \end{table}
        \FloatBarrier

        Suriyeli mültecilerin özellikle kayıtdışı alanda yerli emek gücü ile ikame edilebilir olması,
        yerli işgücünün kayıtlı istihdama geçmesinde etkili olmuştur. Her ne kadar bu etki toplam işgücünde 
        anlamsız çıksa da erkek çalışanlar için kayıtlı istihdamda yer alma oranı \%0,03 civarında artmıştır 
        \footnote{Makalede yerel emek piyasasında kayıtlı istihdamda yer alma oranı istatiksel olarak anlamlıdır.}. 
        Bu oran kadınlar için ise anlamsızdır. Yine bu durum kadınların emek piyasasından ayrılması ile açıklanabilir. 
        Düşük eğitimli çalışanlarda ise bu oran \%0,08 civarındadır.
        Yüksek eğitimli grup ise bundan etkilenmemiştir.


        \FloatBarrier
        \begin{table}[h]
            \centering
            \caption{Kayıtlı İstihdam Oranı İçin Regresyon Sonuçları}
            \begin{tabular}{|ccccccc|}
                \hline
                    & Toplam & Toplam & Erkek & Kadın & Düşük E. & Yüksek E. \\ \hline
                    Mülteci Etkisi (RXT) & 0.00380 & 0.00347 & 0.00825* & -0.00184 & 0.00858*** & -0.00908    \\ 
                    & (0.00226) & (0.00227) & (0.00399) & (0.00202) & (0.00226) & (0.00618)    \\ 
                    Log(Toplam Ticaret) & - & + & + & + & + & + \\ 
                    Düzeltilmiş Yıl Etkisi & + & + & + & + & + & + \\ 
                    Düzeltilmiş Bölge Etkisi & + & + & + & + & + & + \\ 
                    Diğer Kontroller & + & + & + & + & + & + \\
                    Sabit Terim & -0.0751*** & -0.478*** & -0.553* & -0.116 & -0.301* & -0.894**  \\ 
                    & (0.00188) & (0.124) & (0.219) & (0.108) & (0.123) & (0.345)    \\ 
                    R & 0.305 & 0.305 & 0.273 & 0.218 & 0.176 & 0.247    \\ 
                    Gözlem Sayısı & 354.023 & 354.023 & 171.017 & 183.006 & 270.608 & 83.415 \\ \hline
                    \multicolumn{7}{|c|}{\scriptsize Güçlü standart hatalar parantez içinde verilmiştir.   * p<0.05, ** p<0.01, *** p<0.001}\\ \hline
                    \multicolumn{7}{|c|}{\scriptsize Kontrol değişkeni olarak cinsiyet, medeni durum, yaş ve eğitim kukla değişkenleri, yaş ve eğitim değişkenlerinin etkileşim terimleri} \\ 
                    \multicolumn{7}{|c|}{\scriptsize ayrıca bölgenin şehirleşme durumunu gösteren kukla değişken kullanılmıştır.} \\ \hline
                    \multicolumn{7}{|c|}{\scriptsize Yüksek eğitim lise ve üstü eğitimi, düşük eğitim ise lise altı eğitimi ifade etmektedir.} \\ \hline

                \end{tabular}
        \end{table}
        \FloatBarrier

        \newpage

        Tablo 7, müdahale grubunda Suriyeli mültecilerden dolayı işten ayrılma oranlarında nasıl bir 
        değişim yaşandığını göstermektedir. Beklenilenilenin aksine işten ayrılma olasılığı \%0,04 gibi düşük bir oranda 
        olsa da artış göstermiştir
        \footnote{Makalede işten ayrılma olasılıkları her grupta anlamsız çıkmıştır.}. 
        İşten ayrılma olasılığı erkeklerde \%0,05 artış gösterirken kadınlarda bu oran anlamsızdır.
        Düşük eğitim grubunda işten ayrılma olasığı \%0,06 artarken, yüksek eğitim grubunda herhangi bir etki görülmemektedir. 


        \FloatBarrier
        \begin{table}[h]
            \centering
            \caption{İşten Ayrılma Olasılığı İçin Regresyon Sonuçları}
            \begin{tabular}{|ccccccc|}
                \hline
                     & Toplam & Toplam & Erkek & Kadın & Düşük E. & Yüksek E. \\ \hline
                     Mülteci Etkisi (RXT) & 0.00469* & 0.00459* & 0.00504* & 0.00299 & 0.00674** & -0.000107 \\ 
                     & (0.00190) & (0.00189) & (0.00239) & (0.00288) & (0.00233) & (0.00327) \\ 
                    Log(Toplam Ticaret) & - & + & + & + & + & + \\ 
                    Düzeltilmiş Yıl Etkisi & + & + & + & + & + & + \\ 
                    Düzeltilmiş Bölge Etkisi & + & + & + & + & + & + \\ 
                    Diğer Kontroller & + & + & + & + & + & + \\
                    Sabit Terim & 0.0545*** & -0.153 & -0.204 & 0.0255 & -0.147 & -0.175 \\ 
                    & (0.00293) & (0.105) & (0.132) & (0.147) & (0.130) & (0.176) \\ 
                    R & 0.016 & 0.016 & 0.014 & 0.029 & 0.018 & 0.017 \\ 
                    Gözlem Sayısı & 136.824 & 136.824 & 100.549 & 36.275 & 93.536 & 43.288 \\ \hline
                    \multicolumn{7}{|c|}{\scriptsize Güçlü standart hatalar parantez içinde verilmiştir.   * p<0.05, ** p<0.01, *** p<0.001}\\ \hline
                    \multicolumn{7}{|c|}{\scriptsize Kontrol değişkeni olarak cinsiyet, medeni durum, yaş ve eğitim kukla değişkenleri, yaş ve eğitim değişkenlerinin etkileşim terimleri} \\ 
                    \multicolumn{7}{|c|}{\scriptsize ayrıca bölgenin şehirleşme durumunu gösteren kukla değişken kullanılmıştır.} \\ \hline
                    \multicolumn{7}{|c|}{\scriptsize Yüksek eğitim lise ve üstü eğitimi, düşük eğitim ise lise altı eğitimi ifade etmektedir.} \\ \hline
                    \multicolumn{7}{|c|}{\scriptsize Örneklem, yerli nüfus içinde referans yılından bir önceki yıl istihdamda olan bireyleri kapsamaktadır.} \\ \hline

                \end{tabular}
        \end{table}
        \FloatBarrier

        Mültecilerin yerel emek piyasasında iş bulma oranlarını nasıl etkilediği Tablo 8'de görülmektedir.
        Buna göre toplam yerli işgücünde iş bulma olasılığı \%4,5 civarında azalmıştır. Bu oran erkekler için
        \%4,2 düşüş gösterirken kadınlar için anlamsız çıkmıştır. Düşük eğitimli bireylerde ise iş bulma olasılığı
        \%6 civarında azalmıştır. Yüksek eğitimli bireylerde ise herhangi bir etki görülmemiştir.

        \FloatBarrier
        \begin{table}[h]
            \centering
            \caption{İş Bulma Olasılığı İçin Regresyon Sonuçları}
            \begin{tabular}{|ccccccc|}
                \hline
                     & Toplam & Toplam & Erkek & Kadın & Düşük E. & Yüksek E. \\ \hline
                     Mülteci Etkisi (RXT) & -0.0460*** & -0.0458*** & -0.0427** & -0.0615 & -0.0661*** & 0.0157 \\ 
                     & (0.0122) & (0.0122) & (0.0130) & (0.0352) & (0.0143) & (0.0239) \\
                    Log(Toplam Ticaret) & - & + & + & + & + & + \\ 
                    Düzeltilmiş Yıl Etkisi & + & + & + & + & + & + \\ 
                    Düzeltilmiş Bölge Etkisi & + & + & + & + & + & + \\ 
                    Diğer Kontroller & + & + & + & + & + & + \\
                    Sabit Terim & 0.358*** & 0.449 & 0.349 & 3417 & 0.136 & 0.388 \\ 
                    & (0.0131) & (0.663) & (0.697) & -2157 & (0.765) & -1353 \\ 
                    R & 0.044 & 0.044 & 0.047 & 0.050 & 0.044 & 0.055 \\ 
                    Gözlem Sayısı & 29.330 & 29.330 & 25.958 & 3.372 & 21.958 & 7.372 \\ \hline
                    \multicolumn{7}{|c|}{\scriptsize Güçlü standart hatalar parantez içinde verilmiştir.   * p<0.05, ** p<0.01, *** p<0.001}\\ \hline
                    \multicolumn{7}{|c|}{\scriptsize Kontrol değişkeni olarak cinsiyet, medeni durum, yaş ve eğitim kukla değişkenleri, yaş ve eğitim değişkenlerinin etkileşim terimleri} \\ 
                    \multicolumn{7}{|c|}{\scriptsize ayrıca bölgenin şehirleşme durumunu gösteren kukla değişken kullanılmıştır.} \\ \hline
                    \multicolumn{7}{|c|}{\scriptsize Yüksek eğitim lise ve üstü eğitimi, düşük eğitim ise lise altı eğitimi ifade etmektedir.} \\ \hline
                    \multicolumn{7}{|c|}{\scriptsize Örneklem, yerli nüfus içinde referans yılından bir önceki yıl istihdamda yer almayan bireyleri kapsamaktadır.} \\ \hline

                \end{tabular}
        \end{table}
        \FloatBarrier

        \newpage

        Mülteci etkisinin yaş gruplarına göre emek piyasasında nasıl bir etki yarattığı Tablo 9'da özet 
        halinde gösterilmiştir. Özellikle 35 yaş altı bireyler kayıtdışı işlerini kaybetmiş, işgücünden ayrılmış ya da 
        işsiz kalmışlardır. 
        \FloatBarrier
        \begin{table}[h]
            \centering
            \caption{Yaş Gruplarına Göre Mülteci Etkisi}
            \begin{tabular}{|ccccccc|}
                \hline
                    
                & 15-24 & 25-34 & 35-44 & 45-54 & 55-64 & Toplam \\ \hline
                LFP & -0.00456 & -0.0233*** & -0.00916 & -0.0112 & -0.00632 & -0.0117*** \\ 
                 & (0.00544) & (0.00546) & (0.00573) & (0.00705) & (0.0101) & (0.00283)    \\ 
                U/P & 0.0114*** & 0.0162*** & 0.00526 & -0.00326 & 0.00215 & 0.00810*** \\ 
                 & (0.00275) & (0.00339) & (0.00306) & (0.00296) & (0.00255) & (0.00140)    \\ 
                IE/P & -0.0199*** & -0.0264*** & -0.0280*** & -0.00716 & -0.0203* & -0.0224*** \\ 
                 & (0.00486) & (0.00570) & (0.00643) & (0.00742) & (0.00958) & (0.00280)    \\ 
                FE/P & 0.00379 & -0.0113* & 0.0148** & 0.000458 & 0.0131* & 0.00347 \\ 
                 & (0.00303) & (0.00536) & (0.00551) & (0.00613) & (0.00645) & (0.00227) \\ 
                SP & 0.00717 & 0.00870* & 0.00337 & 0.000267 & 0.00127 & 0.00459* \\ 
                 & (0.00638) & (0.00398) & (0.00332) & (0.00340) & (0.00395) & (0.00189) \\ 
                JFP & -0.0671** & -0.0292 & -0.0218 & -0.0387 & -0.0438 & -0.0458*** \\ 
                 & (0.0208) & (0.0219) & (0.0286) & (0.0364) & (0.0689) & (0.0122)    \\ \hline
                   
                    \multicolumn{7}{|c|}{\scriptsize Güçlü standart hatalar parantez içinde verilmiştir.   * p<0.05, ** p<0.01, *** p<0.001}\\ \hline
                    
                \end{tabular}
        \end{table}
        \FloatBarrier

        Mülteci akımının kayıtlı çalışanların ücretleri üzerinde nasıl bir etki yarattığı Tablo 10'dan görülebilir.
        Analiz sonuçlarına göre mülteci akımı kayıtlı çalışanların ücretlerine, yüksek eğitimli bireyler haricinde
        herhangi bir etkide bulunmamıştır. Yüksek eğitimli çalışanların ücretlerindeki \%2'lik artış,
        müdahale bölgesinde mülteciler için yürütülen faaliyetlerle ilişkilendirilebilir.

        \FloatBarrier
        \begin{table}[h]
            \centering
            \caption{Kayıtlı Çalışan Aylık Ücret Geliri için Regresyon Sonuçları}
            \begin{tabular}{|ccccccc|}
                \hline
                     & Toplam & Toplam & Erkek & Kadın & Düşük E. & Yüksek E. \\ \hline
                     Mülteci Etkisi (RXT) & 0.0119 & 0.0141 & 0.0143 & 0.0162 & -0.00181 & 0.0262*   \\ 
                     & (0.00707) & (0.00811) & (0.00889) & (0.0191) & (0.0116) & (0.0109)    \\
                    Log(Toplam Ticaret) & - & + & + & + & + & + \\ 
                    Düzeltilmiş Yıl Etkisi & + & + & + & + & + & + \\ 
                    Düzeltilmiş Bölge Etkisi & + & + & + & + & + & + \\ 
                    Diğer Kontroller & + & + & + & + & + & + \\
                    Sabit Terim & 5.624*** & 5.079*** & 4.799*** & 6.429*** & 3.925*** & 5.577*** \\ 
                    & (0.0333) & (0.446) & (0.488) & -1066 & (0.605) & (0.603)    \\ 
                    R & 0.563 & 0.428 & 0.427 & 0.473 & 0.338 & 0.305    \\ 
                    Gözlem Sayısı & 52.689 & 52.689 & 42.928 & 9.761 & 20.606 & 32.083 \\ \hline
                    \multicolumn{7}{|c|}{\scriptsize Güçlü standart hatalar parantez içinde verilmiştir.   * p<0.05, ** p<0.01, *** p<0.001}\\ \hline
                    \multicolumn{7}{|c|}{\scriptsize Kontrol değişkeni olarak cinsiyet, medeni durum, yaş ve eğitim kukla değişkenleri, yaş ve eğitim değişkenlerinin etkileşim terimleri} \\ 
                    \multicolumn{7}{|c|}{\scriptsize ayrıca bölgenin şehirleşme durumunu gösteren kukla, bireylerin full-time çalışıp çalışmadığını gösteren kukla değişken,} \\
                    \multicolumn{7}{|c|}{\scriptsize firma büyüklüğü kukla değişkenleri ve sektör kukla değişkenleri kullanılmıştır. } \\ \hline
                    \multicolumn{7}{|c|}{\scriptsize Yüksek eğitim lise ve üstü eğitimi, düşük eğitim ise lise altı eğitimi ifade etmektedir.} \\ \hline
                    \multicolumn{7}{|c|}{\scriptsize Örneklem, yerli nüfus içinde referans yılında kayıtlı istihdamda yer alan ücretli çalışanları kapsamaktadır.} \\ \hline

                \end{tabular}
        \end{table}
        \FloatBarrier    









    \end{justify}




\end{document}